%% Generated by Sphinx.
\def\sphinxdocclass{report}
\documentclass[letterpaper,10pt,english]{sphinxmanual}
\ifdefined\pdfpxdimen
   \let\sphinxpxdimen\pdfpxdimen\else\newdimen\sphinxpxdimen
\fi \sphinxpxdimen=.75bp\relax

\PassOptionsToPackage{warn}{textcomp}
\usepackage[utf8]{inputenc}
\ifdefined\DeclareUnicodeCharacter
% support both utf8 and utf8x syntaxes
  \ifdefined\DeclareUnicodeCharacterAsOptional
    \def\sphinxDUC#1{\DeclareUnicodeCharacter{"#1}}
  \else
    \let\sphinxDUC\DeclareUnicodeCharacter
  \fi
  \sphinxDUC{00A0}{\nobreakspace}
  \sphinxDUC{2500}{\sphinxunichar{2500}}
  \sphinxDUC{2502}{\sphinxunichar{2502}}
  \sphinxDUC{2514}{\sphinxunichar{2514}}
  \sphinxDUC{251C}{\sphinxunichar{251C}}
  \sphinxDUC{2572}{\textbackslash}
\fi
\usepackage{cmap}
\usepackage[T1]{fontenc}
\usepackage{amsmath,amssymb,amstext}
\usepackage{babel}



\usepackage{times}
\expandafter\ifx\csname T@LGR\endcsname\relax
\else
% LGR was declared as font encoding
  \substitutefont{LGR}{\rmdefault}{cmr}
  \substitutefont{LGR}{\sfdefault}{cmss}
  \substitutefont{LGR}{\ttdefault}{cmtt}
\fi
\expandafter\ifx\csname T@X2\endcsname\relax
  \expandafter\ifx\csname T@T2A\endcsname\relax
  \else
  % T2A was declared as font encoding
    \substitutefont{T2A}{\rmdefault}{cmr}
    \substitutefont{T2A}{\sfdefault}{cmss}
    \substitutefont{T2A}{\ttdefault}{cmtt}
  \fi
\else
% X2 was declared as font encoding
  \substitutefont{X2}{\rmdefault}{cmr}
  \substitutefont{X2}{\sfdefault}{cmss}
  \substitutefont{X2}{\ttdefault}{cmtt}
\fi


\usepackage[Bjarne]{fncychap}
\usepackage{sphinx}

\fvset{fontsize=\small}
\usepackage{geometry}


% Include hyperref last.
\usepackage{hyperref}
% Fix anchor placement for figures with captions.
\usepackage{hypcap}% it must be loaded after hyperref.
% Set up styles of URL: it should be placed after hyperref.
\urlstyle{same}
\addto\captionsenglish{\renewcommand{\contentsname}{Contents:}}

\usepackage{sphinxmessages}
\setcounter{tocdepth}{1}



\title{Marine user guide}
\date{Jun 24, 2020}
\release{v1.0}
\author{IPG}
\newcommand{\sphinxlogo}{\vbox{}}
\renewcommand{\releasename}{Release}
\makeindex
\begin{document}

\pagestyle{empty}
\sphinxmaketitle
\pagestyle{plain}
\sphinxtableofcontents
\pagestyle{normal}
\phantomsection\label{\detokenize{index::doc}}



\chapter{Indices and tables}
\label{\detokenize{index:indices-and-tables}}\begin{itemize}
\item {} 
\DUrole{xref,std,std-ref}{genindex}

\item {} 
\DUrole{xref,std,std-ref}{modindex}

\item {} 
\DUrole{xref,std,std-ref}{search}

\end{itemize}


\chapter{Introduction}
\label{\detokenize{index:introduction}}
This project contains the necessary code to produce the data summaries that are
included in the Marine User Guide. These help to document the status of the
marine in situ data in the CDS after every new data release.

The marine data available in the CDS is the result of a series of data releases
that are stored in the marine data file system in different directories. This
project uses the data in the marine file system, rather than accessing to the
CDS data.

Every new data release can potentially be created with a different version of
the marine processing software. The current version of this project is
compatible with the glamod\sphinxhyphen{}marine\sphinxhyphen{}processing code up to version v1.2.


\chapter{Initialize a new user guide}
\label{\detokenize{index:initialize-a-new-user-guide}}
To initialise a new version of the Marine User Guide:

1. Create the data configuration (\sphinxstyleemphasis{mug\_file}) file by merging the level2  configuration files of the different data releases in the current User Guide  version.
\begin{quote}

\begin{sphinxVerbatim}[commandchars=\\\{\}]
python \PYGZlt{}mug\PYGZgt{}/init\PYGZus{}version/init\PYGZus{}config.py
\end{sphinxVerbatim}
\end{quote}

2. Use the sid\sphinxhyphen{}dck keys of the \sphinxstyleemphasis{mug\_config} file to create a simple ascii file  with the list of sid\sphinxhyphen{}dcks to process (\sphinxstyleemphasis{mug\_list}).

3. Create a view of the merged data releases in the User Guide data subspace:  this is done by linking the level2 data from the releases to the same level2  subdirectory in user manual subspace. The launcher script also initialises the subdirectory for the user guide version in the data user guide space.
\begin{quote}

\begin{sphinxVerbatim}[commandchars=\\\{\}]
./\PYGZlt{}mug\PYGZgt{}/init\PYGZus{}version/merge\PYGZus{}release\PYGZus{}data\PYGZus{}launcher.sh *version* *mug\PYGZus{}config* *mug\PYGZus{}list*
\end{sphinxVerbatim}

\begin{figure}[htbp]
\centering

\noindent\sphinxincludegraphics[width=300\sphinxpxdimen]{{file_links}.png}
\end{figure}
\end{quote}


\chapter{Data summaries}
\label{\detokenize{index:data-summaries}}\begin{enumerate}
\sphinxsetlistlabels{\arabic}{enumi}{enumii}{}{.}%
\item {} 
Monthly grids

\end{enumerate}

\begin{sphinxVerbatim}[commandchars=\\\{\}]
./\PYGZlt{}mug\PYGZgt{}/data\PYGZus{}summaries/monthly\PYGZus{}grids\PYGZus{}launcher.sh \PYG{l+s+se}{\PYGZbs{}\PYGZbs{}}*version\PYG{l+s+se}{\PYGZbs{}\PYGZbs{}}* \PYG{l+s+se}{\PYGZbs{}}
\PYGZlt{}glamod\PYGZhy{}marine\PYGZhy{}config\PYGZgt{}/marine\PYGZhy{}user\PYGZhy{}guide/*version*/monthly\PYGZus{}grids.json
\end{sphinxVerbatim}
\begin{enumerate}
\sphinxsetlistlabels{\arabic}{enumi}{enumii}{}{.}%
\setcounter{enumi}{1}
\item {} 
Quality indicators time series summaries

\end{enumerate}



\renewcommand{\indexname}{Index}
\printindex
\end{document}